\documentclass{article}
\usepackage{graphicx}
\begin{document}
\title{Neuroinformatics with the Insight ToolKit}
\author{Brian B. Avants et al}
\maketitle
\begin{abstract}
A 500 word summary of our topic. 
\end{abstract}
The Insight ToolKit (ITK) is a multi-platform open-source image analysis library that may be used freely within both academia and industry.  The ITK software project was initiated in 1999 by the United States National Library of Medicine (NLM) to address the pressing need for a robust multi-dimensional and multiple modality image analysis platform.  In 2010, NLM and collaborators refactored ITK such that its capabilities and software foundation would be sustainable for the next 10 years.  This effort led to ITKv4 which is currently in its 4.3 release.  

The primary goal of this proposal is to highlight and detail ITKv4's segmentation and registration methodology and its application to a variety of neuroinformatics problems.  By using ITKv4 as a core platform and disseminating the analysis pipelines, the papers selected by this proposal will promote reproducible practices and act as a guide for other researchers who wish to use ITKv4 in their own research.  

Papers will be selected from a broad range of neuroinformatics research areas.  Applications will include basic methodology in addition to modeling cellular level growth, large-scale atrophy in neurodegeneration, physiological motion and surgical simulation.  These selections will illustrate how open-source software may be used to improve the quality of patient care, help illuminate fundamental biological processes or reveal high-level relationships between brain and behavior.  

Each contributed paper will:
\begin{itemize}
\item  Describe and motivate the targeted application area.
\item  Detail the theoretical and/or software framework that will address the application area.
\item  Establish a well-defined evaluation goal. 
\item  Provide access to a public reference dataset.  A subset of results should be based on this dataset.
\item  Share software that is applicable to the reference dataset and which addresses the evaluation/research goal.  
\item  Summarize the unique contributions of the work.  
\end{itemize}
Submissions are encouraged from both software developers and scientists who use ITK or ITK-derived software in their research.

\paragraph{Abstract Submission deadline:} February 15, 2013 (1 page abstract). 

\paragraph{Article Submission deadline:} June 15, 2013 (full article submission).

\end{document}

